\documentclass[]{article}
\usepackage{lmodern}
\usepackage{amssymb,amsmath}
\usepackage{ifxetex,ifluatex}
\usepackage{fixltx2e} % provides \textsubscript
\ifnum 0\ifxetex 1\fi\ifluatex 1\fi=0 % if pdftex
  \usepackage[T1]{fontenc}
  \usepackage[utf8]{inputenc}
\else % if luatex or xelatex
  \ifxetex
    \usepackage{mathspec}
  \else
    \usepackage{fontspec}
  \fi
  \defaultfontfeatures{Ligatures=TeX,Scale=MatchLowercase}
\fi
% use upquote if available, for straight quotes in verbatim environments
\IfFileExists{upquote.sty}{\usepackage{upquote}}{}
% use microtype if available
\IfFileExists{microtype.sty}{%
\usepackage{microtype}
\UseMicrotypeSet[protrusion]{basicmath} % disable protrusion for tt fonts
}{}
\usepackage[margin=1in]{geometry}
\usepackage{hyperref}
\hypersetup{unicode=true,
            pdfborder={0 0 0},
            breaklinks=true}
\urlstyle{same}  % don't use monospace font for urls
\usepackage{graphicx,grffile}
\makeatletter
\def\maxwidth{\ifdim\Gin@nat@width>\linewidth\linewidth\else\Gin@nat@width\fi}
\def\maxheight{\ifdim\Gin@nat@height>\textheight\textheight\else\Gin@nat@height\fi}
\makeatother
% Scale images if necessary, so that they will not overflow the page
% margins by default, and it is still possible to overwrite the defaults
% using explicit options in \includegraphics[width, height, ...]{}
\setkeys{Gin}{width=\maxwidth,height=\maxheight,keepaspectratio}
\IfFileExists{parskip.sty}{%
\usepackage{parskip}
}{% else
\setlength{\parindent}{0pt}
\setlength{\parskip}{6pt plus 2pt minus 1pt}
}
\setlength{\emergencystretch}{3em}  % prevent overfull lines
\providecommand{\tightlist}{%
  \setlength{\itemsep}{0pt}\setlength{\parskip}{0pt}}
\setcounter{secnumdepth}{0}
% Redefines (sub)paragraphs to behave more like sections
\ifx\paragraph\undefined\else
\let\oldparagraph\paragraph
\renewcommand{\paragraph}[1]{\oldparagraph{#1}\mbox{}}
\fi
\ifx\subparagraph\undefined\else
\let\oldsubparagraph\subparagraph
\renewcommand{\subparagraph}[1]{\oldsubparagraph{#1}\mbox{}}
\fi

%%% Use protect on footnotes to avoid problems with footnotes in titles
\let\rmarkdownfootnote\footnote%
\def\footnote{\protect\rmarkdownfootnote}

%%% Change title format to be more compact
\usepackage{titling}

% Create subtitle command for use in maketitle
\providecommand{\subtitle}[1]{
  \posttitle{
    \begin{center}\large#1\end{center}
    }
}

\setlength{\droptitle}{-2em}

  \title{}
    \pretitle{\vspace{\droptitle}}
  \posttitle{}
    \author{}
    \preauthor{}\postauthor{}
    \date{}
    \predate{}\postdate{}
  

\begin{document}

\hypertarget{week-2-exercises}{%
\section{Week 2 Exercises}\label{week-2-exercises}}

install.packages(``dplyr'')

install.packages(``ggplot2'')

install.packages(``GGally'')

library(dplyr)

library(ggplot2)

library(GGally)

\hypertarget{reading-the-data-and-describing-it}{%
\subsection{1. Reading the data and describing
it}\label{reading-the-data-and-describing-it}}

analysisData \textless{}- read.table(`data/learning2014')
summary(analysisData\$Points)

``This data is collected from the social science students of the
Univeristy of Helsinki. The survey was conducted in a social sciences
statistical course for bachelor students in 2014 December to 2015
January. The survey consists of 7 variables that describe the gender
(factored as female or male), age, attitude (towards the study of
statistics), deep learning (pupils own attitudes and practices towards
deep learning), surface learning and strategic learning as well as their
received points in the final exam of the course. The learning scales
mostly try to get the student to show what kind of strategies and types
of learning they try to implement in their studies. All the learning
categories are number values in likert-scales (values from 1 to 5).
Points, attitude and age are integer values. The survey was completed in
Finnish.''

\hypertarget{graphical-overview-of-the-data}{%
\subsection{2. Graphical overview of the
data}\label{graphical-overview-of-the-data}}

ggpairs(analysisData, lower = list(combo = wrap(``facethist'', bins =
20)))

\hypertarget{correlation-between-variables}{%
\subsubsection{Correlation between
variables}\label{correlation-between-variables}}

``In out pairs plot we can see that the correlations between the
variables are quite low. Only a real significant correlation can be
found between attitude and points (0,44) and deep learning strategies
and surface level strategies (the correlation between the two strategies
is negative).''

\hypertarget{distributions-of-variances}{%
\subsubsection{Distributions of
variances}\label{distributions-of-variances}}

``Almost all of the variances of the variables are noramlly
distributed.Tge onyl exception is age, which is explained due to the
fact that the participants of the survey are students, who are young and
mostly the same age.''

\hypertarget{distribution-of-residuals}{%
\subsubsection{Distribution of
residuals}\label{distribution-of-residuals}}

``In most of the cases we cannot see a clear linear pattern in the
residuals between two variables. Some show a clear pattern, like
attitude and surface learning strategies, where the residuals are skewed
towards negative values.''

\hypertarget{own-linear-regression-model-4.-summary-of-the-fitted-model}{%
\subsection{3. Own linear regression model \& 4. Summary of the fitted
model}\label{own-linear-regression-model-4.-summary-of-the-fitted-model}}

myRegressionModel \textless{}- lm(Points \textasciitilde{} Attitude +
stra + surf, data = analysisData) summary(myRegressionModel)

``Here i created a linear regression model where the students attitude,
strategic learning techniques and surface techniques try to explain the
students achievement levels in the final exam of the course. In this
first model, we can see that surface and stategic learning do not have
any significance on the students outcomes in the course test. Their
attitude towards the study of statistics, however, proves to be very
significant.''

secondModel \textless{}- lm(Points \textasciitilde{} Attitude, data =
analysisData) summary(secondModel)

``In the second model, we can see that the students attitudes have an
extremly significant causal relation to the points that they receive in
the final exam. This means that the likelyhood to get a dataset this
deviant is less than one percent. In this model one point in attitude
increased the points that a student got in the exam by 0.35. The
multiple R-squared means that the variance of the points in the test can
be explained by about 19 percent by the differences in the attitudes of
the students.''

\hypertarget{diagnostic-plots}{%
\subsection{5. Diagnostic plots}\label{diagnostic-plots}}

par(mfrow = c(2,2)) plot(x=secondModel, which= c(1,2,5))

``In the assumptions of the model, we assume the variables are normally
distributed and that their variance is constant. From the residuals vs
fitted, we can see that there seems to be no clear pattern in the
distribution of the variables and thefore we can conclude that the model
is quite linear. Also the variance is seems to be quite constant as the
datapoints dont seem to stray too much from the mean. The In the
normalQQ- plot we can see that our data pretty much follows the line.
This means that the data is normally distributed. In the residuals and
leverage plot we see that there arent really any outliers that have a
big leverage on the rest of the data. The data is bit skewed down, but
not significantly.''


\end{document}
